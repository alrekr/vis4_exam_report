%!TEX root = vis4_report.tex
\autsection{Introduction}{Malthe Høj-Sunesen}
According to ISO 8373~\cite{detry}, at least two different types robots exist:
Industrial robots and service robots.
An industrial robot is defined as a ``automatically controlled, reprogrammable, multipurpose manipulator programmable in three or more axes'',
while a service robot is defined as a ``robot that performs useful tasks for humans or equipment excluding industrial automation applications''.
The classical application of an industrial robot is to have the robot do a predefined behavior repeatedly,
while service robots are still very much under development.
Due to hardware and software concerns, robots in the industry have previously not seen adaptive behavior,
so elements must be aligned in a specific way. Humans are able to look at objects and behave accordingly.
A lot of research is going into making the robot able to understand what it is ``looking'' at much like humans can,
and how to grasp it. This research into grasping objects using only visual cues is the focus point for this report.

For the purposes of this report, \emph{grasping} is to pick up an object.
